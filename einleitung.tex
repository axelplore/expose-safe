
% !TeX root = main.tex

\section{Einleitung} 
\label{sec:einleitung}

Die Studie von Dwivedi et al.\footnote{Vgl. \cite{DwivediHughes2021}, S.11-13} beschreibt KI als eine transformative Technologie, die menschliche Aufgaben und Aktivitäten in vielen Bereichen ergänzen oder ersetzen kann. Ein besonders vielversprechender Aspekt in diesem Bereich ist die Anwendung von Natural Language Processing (NLP) in betrieblichen bzw. administrativen Arbeitsprozessen. Die Anforderungen an die Arbeitnehmer werden sich wandeln. Es werden neue Fähigkeiten und Kompetenzen erforderlich sein, zu denen auch Problemlösungs- und IT-Fähigkeiten zählen.\footnote{Vgl. \cite{Kadir2019}, S.9-10}
Die Einzelhandelsbranche, insbesondere der Schuhsektor, bietet vielseitige Gelegenheiten, diese Vorteile zu nutzen. Manuelle Prozesse im Einzelhandel, wie z. B. Bestandsverwaltung, Kundenservice und Verkaufsanalyse, sind oft zeitaufwändig und fehleranfällig.\footnote{Vgl. \cite{Perez2018}, S.1-10; \cite{Lee2018}, S.20} NLP in Arbeitsprozessen hat das Potenzial, die Fähigkeiten der Belegschaft zu erweitern und die Produktivität zu erhöhen, indem sie Routineaufgaben automatisiert, fortschrittliche Analysen ermöglicht und personalisiertes Coaching bereitstellt.\footnote{Vgl. \cite{Tasheva2024}, S.26-27} Durch die Kombination menschlicher Stärken mit KI-Technologien kann ein höheres Maß an Innovation, Effizienz und Leistung erreicht werden.

Mit der Veröffentlichung des NLP-Modell GPT-3 im Jahr 2020 hat OpenAI einen Meilenstein in der KI-Forschung gesetzt. GPT-3 ist ein Sprachmodell, das auf der Grundlage von 175 Milliarden Parametern trainiert wurde und in der Lage ist, menschenähnliche Texte zu generieren.\footnote{Vgl. \cite{Brown2020}, S.1878} NLP hat das Potenzial, die Art und Weise, wie wir mit Computern interagieren, zu revolutionieren und die KI-Technologie auf ein neues Niveau zu heben.\footnote{Vgl. \cite{Lu2021}, S.1046-1047} OpenAI hat das Modell als Chatbot kostenlos und für jedermann zugänglich gemacht, wodurch die KI-Technologie in die Gesellschaft eingeführt wurde. Erste Studien haben das Potenzial von NLP für die Umgestaltung von Geschäftsprozessen aufgezeigt. So kann laut einer Studie von Prabhavathi et al.\footnote{Vgl. \cite{Prabhavathi2019}, S.161-162} die Integration von NLP im Einzelhandel das Kundenerlebnis durch die Automatisierung von Abfrageantworten und die Bereitstellung personalisierter Empfehlungen erheblich verbessern. In ähnlicher Weise zeigten Ibrahima et. al.\footnote{Vgl. \cite{Ibrahima2021}, S.33}, wie NLP-gestützte Business Intelligence-Lösungen, die Genauigkeit der Bestandsverwaltung durch die Analyse von Verkaufsmustern und die Vorhersage des Lagerbedarfs verbessern können. In einer gemeinsamen Studie von Stanford und MIT wurde festgestellt, dass der Einsatz von KI-Tools wie Chatbots die Produktivität der Mitarbeiter in einem Technologieunternehmen um 14\% erhöhte.\footnote{Vgl. \cite{Brynjolfsson2023}, S.2}

Die Entwicklung von Tools wie Microsofts Copilot für Microsoft 365 zeigt die praktische Anwendung von NLP bei der Automatisierung von Routineaufgaben und der Steigerung der Produktivität im Unternehmenskontext.\footnote{Vgl. \cite{Spataro2024}, S.1-7} Microsoft Copilot ist ein KI-gestütztes Werkzeug zur Produktivitätssteigerung, das sich nahtlos in Microsoft 365-Anwendungen integrieren lässt. Es unterstützt Nutzer dabei, ihre Kreativität zu entfalten, die Produktivität zu erhöhen und Fähigkeiten zu erweitern. Copilot vereint die Leistungsfähigkeit von Large Language Models (LLM) mit Benutzerdaten aus Microsoft Graph und Microsoft 365-Anwendungen, um Benutzereingaben in ein effektives Produktivitätswerkzeug zu verwandeln. Es hilft den Menschen, effizienter zu arbeiten und bessere Ergebnisse zu erzielen. Zudem ist Copilot flexibel und vollständig in beliebte Microsoft-Anwendungen wie Teams, Outlook, Word und Excel integriert.\footnote{Vgl. \cite{Vasilescu2024}, S.2-3}

Die Relevanz dieses Themas wird durch eine Analyse von Bughin et. al.\footnote{Vgl. \cite{Bughin2018}, S.3} unterstrichen. Laut der Analyse könnte die intelligente Automatisierung und der Einsatz von KI die globale Produktivität jährlich um 0,8 bis 1,4\% erhöhen. Es wird prognostiziert, dass die Weltwirtschaft bis 2030 durch den Einsatz von KI bis zu 15,7 Billionen Dollar gewinnen könnte, hauptsächlich weil KI die Arbeitskräfte ergänzt, anstatt sie zu ersetzen.\footnote{Vgl. \cite{PWC2017}, S.5}


\newpage