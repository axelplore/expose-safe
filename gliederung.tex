% !TeX root=main.tex

\section{Vorläufige Gliederung der Bachelor-Thesis}

Die Thesis wurde in 7 Kapitel unterteilt und wurde insgesamt für eine ungefähre Seitenanzahl von 40-45 Seiten geplant. Die Gliederung der Thesis ist in Tabelle \ref{tab:gliederung} dargestellt.

\begin{longtable}{|p{10cm}|p{2.5cm}|}
    \caption{Gliederung der Bachelorthesis} \label{tab:gliederung} \\
    \hline
    \textbf{Kapitel} & \textbf{Seitenanzahl} \\
    \hline
    I.      Abbildungsverzeichnis & \\
    \hline
    II.     Abkürzungsverzeichnis & \\
    \hline
    III.    Formelverzeichnis & \\
    \hline
    IV.     Tabellenverzeichnis & \\
    \hline
    \textbf{1. Einleitung} & \textbf{6 Seiten} \\
    \hline
        1.1. Motivation und Problemstellung & 1 Seite \\
        \hline
        1.2. Zielsetzung und Forschungsfrage & 3 Seite \\
        \hline
        1.3. Untersuchungsgegenstand & 1 Seite \\
        \hline
        1.4. Aufbau der Arbeit & 1 Seite \\
    \hline
    \textbf{2. Theoretischer Rahmen} & \textbf{7 Seiten} \\
    \hline
        2.1. Literaturrecherche & 2 Seite \\
        \hline
        2.2. NLP \& Microsoft 365 Copilot & 2 Seiten \\
        \hline
        2.3. Produktivität \& Effizienz & 1 Seiten \\
        \hline
        2.4. Aktuelle Herausforderungen und Forschungslücken & 1 Seite \\
        \hline
        2.5. Beschreibung des Unternehmens und des Pilotprojekts & 1 Seite \\
    \hline
    \textbf{3. Methodik} & \textbf{9 Seiten} \\
    \hline
        3.1. Forschungsdesign (Expertenbefragung: Begründung der Wahl) & 5 Seiten \\
        \hline
        3.2. Zielgruppe der Befragung (Beschreibung der Experten) & 2 Seiten \\
        \hline
        3.3. Datenerhebungsmethode (halbstrukturierte Interviews) & 2 Seite \\
    \hline
    \textbf{4. Durchführung der Expertenbefragung} & \textbf{5 Seiten} \\
    \hline
        4.1. Durchführung der Interviews/Umfragen (Wie wurden die Experten kontaktiert? Wie viele haben teilgenommen? Dauer und Umfang der Interviews) & 2 Seite \\
        \hline
        4.2. Fragebogenentwicklung: Welche Schlüsselfragen wurden gestellt?  & 3 Seiten \\
    \hline
    \textbf{5. Analyse der Ergebnisse} & \textbf{3 Seiten} \\
    \hline
        5.1. Datenaufbereitung: Wie wurden die Antworten der Experten kategorisiert und ausgewertet? Verwendung von Kodierungen zur Identifikation zentraler Themen und Muster & 3 Seiten \\

    \hline
    \textbf{6. Diskussion der Ergebnisse \& Ausblick} & \textbf{12 Seiten} \\
    \hline
        6.1. Interpretation der Ergebnisse: Wie beeinflusst Copilot verschiedene Abteilungen oder Prozesse? Gibt es Unterschiede in den Abteilungen? Wurden die Erwartungen der Experten erfüllt? & 3 Seite \\
        \hline
        6.2. Kritische Reflexion der Ergebnisse im Vergleich zur Literatur. Was sind die Vor- und Nachteile des Tools? Gibt es Verbesserungspotenzial? & 3 Seite \\
        \hline
        6.3. Diskussion der Methodik & 2 Seite \\
        \hline
        6.4. Vergleich zu den anderen Branchen & 2 Seite \\
        \hline
        6.5. Praktische Implikationen: Wie könnten Unternehmen diese Tools optimal einsetzen? & 1 Seite \\
        \hline
        6.6. Forschungsperspektiven: Welche offenen Fragen bleiben und wie könnte zukünftige Forschung aussehen? & 1 Seite \\
    \hline
    \textbf{7. Fazit} & \textbf{1 Seite} \\
    \hline
    V. Literaturverzeichnis & \\
    \hline
    VI. Anhang & \\
    \hline
\end{longtable}



\clearpage