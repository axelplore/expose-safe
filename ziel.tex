% !TeX root=main.tex

\section{Forschungsfragen und Ziel der Arbeit}

Um die Forschungsfrage zu entwickeln, musste ich erst verstehen, was eine Forschungsfrage ausmacht bzw. welche Merkmale eine Forschungsfrage besitzt. Gläser und Laudel\footnote{Vgl. \cite{Glaeser2010}, S.65} geben dafür folgende Hilfestellung:
\begin{itemize}
    \item Bereits existierendes Wissen greift Theorien mit deren Begriffe auf und stellt eine bisher noch nicht beantwortete Frage
    \item Die Antwort auf die Frage erzeugt neuen wissenschaftlichen Mehrwehrt
    \item Es wird nach einem Zusammenhang zwischen Bedingungen, Verlauf und Wirkungen von Prozessen gefragt
    \item Es wird nach einem allgemeinen Zusammenhang auf einen Prozesstypen gefragt
\end{itemize}

Der Hintergrund der Untersuchung des M365 Copiloten bildet die rasante Entwicklung von NLP-Modellen in den letzten Jahren und das öffentliche Interesse bzw. das Interesse der Unternehmen diese Technologie in ihre Arbeitsprozesse zu integrieren. \footnote{Vgl. \cite{Syed2020}, S.120-143}

Die zentrale Forschungsfrage dieser Bachelorthesis lautet:

\begin{description}
    \item[MRQ:] ''Wie beeinflusst der Einsatz von KI-basierten Tools wie Microsoft 365 Copilot die Effizienz und Produktivität der administrativen Arbeitsprozesse in Unternehmen des Schuheinzelhandels?''
\end{description}

Diese Frage bildet den Ausgangspunkt der Untersuchung und fokussiert auf die Integration von NLP wie Microsoft 365 Copilot im Arbeitsalltag und dahingehenden Einfluss auf Produktivität und Effizienz. Die Forschungsfrage behandelt die Forschungslücke einer effektiven Mensch-KI Zusammenarbeit und wie der Einsatz von KI-Tools den Arbeitsalltag verbessert oder beeinträchtigt.

Aus der zentralen Forschungsfrage ergeben sich folgende spezifische Forschungsfragen:
\begin{description}
    \item[RQ1:] Es sollen Variablen, die von NLP-Technologien wie Microsoft Copilot in administrativen Arbeitsprozessen profitieren würden, identifiziert werden. Hierfür soll die Fragestellung, die von Vasilescu \& Gheorghe\footnote{Vgl. \cite{Vasilescu2024}, S.1819} entwickelt wurde,\textbf{''Wie hilft Copilot den Mitarbeitern, Aufgaben schneller zu erledigen, Entscheidungen zu treffen, mehr Arbeit zu erledigen?''} beantwortet werden.
    \item[RQ2:] Es soll die Zusammenarbeit von Mensch und KI bei der Arbeit und das menschliche Wohlergehen untersucht werden. Für diesen Fall soll die Fragestellung von Bankins et. al.\footnote{Vgl. \cite{Bankins2024}, S.171} \textbf{''Wie kann KI eingesetzt werden, um das menschliche Wohlergehen bei der Arbeit zu verbessern?''} beantwortet werden.
    \item[RQ3:] Es soll die benötigten Fähigkeiten für die Verwendung von NLP der Arbeitnehmer untersucht werden, die sich mit eine Erhöhung der Produktivität in betrieblichen Arbeitsprozessen auswirken. Hierfür soll die Fragestellung von Bankins et. al.\footnote{Vgl. \cite{Bankins2024}, S.171} \textbf{''Wie kann KI auf die unterschiedlichen Bedürfnisse von Arbeitnehmern mit verschiedenen kognitiven Fähigkeiten zugeschnitten werden, um die Produktivität zu optimieren?''} beantwortet werden. 
    \item[RQ4:] Es sollen die Auswirkungen auf die Prozesse innerhalb von Teams untersucht werden, die ihre Arbeit mithilfe von NLP unterstützen. Hierfür soll die Fragestellung von Bankins et. al.\footnote{Vgl. \cite{Bankins2024}, S.171} \textbf{''Wie werden die Prozesse von Teams durch den Einsatz von KI verbessert oder beeinträchtigt?''} beantwortet werden. 
\end{description}

Das Ziel dieser Bachelorthesis ist es, ein fundiertes Verständnis dafür zu entwickeln, wie LLMs innerhalb von BI-Lösungen genutzt werden können, um manuelle Prozesse zu automatisieren und somit die Effizienz und Produktivität in einem Schuhhandelsunternehmen zu steigern. Durch die Identifikation spezifischer manueller Prozesse und die Implementierung von Microsoft 365 Copilot soll aufgezeigt werden, wie LLMs praktisch eingesetzt werden können, um die Datenanalyse und Berichtserstellung zu verbessern.

Darüber hinaus soll die Arbeit die Auswirkungen der Automatisierung auf die Genauigkeit der Datenverarbeitung und die Produktivität der Mitarbeiter untersuchen, um fundierte Empfehlungen für die weitere Integration von LLMs in BI-Systeme zu geben. Durch die Auseinandersetzung mit den Herausforderungen und Grenzen der Implementierung wird zudem ein umfassender Überblick über die praktischen Implikationen der Nutzung von LLMs im BI-Kontext vermittelt.

\clearpage