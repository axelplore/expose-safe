
% !TeX root=main.tex

\section{Arbeitstitel}

Der Arbeitstitel für diese Bachelorthesis lautet: ''Einfluss von Natural-Language-Processing-Tools wie Microsoft 365 Copilot auf betriebliche Arbeitsprozesse im Schuheinzelhandel: Eine Expertenbefragung''. Dieser Titel wurde gewählt, um die zentrale Fragestellung der Arbeit prägnant und klar zu kommunizieren. Im Fokus stehen dabei die Nutzung von NLP zur Automatisierung verschiedener manueller Prozesse innerhalb von betrieblichen Arbeitsprozessen und die damit verbundenen Einfluss. 

Der Titel wurde nach sorgfältiger Überlegung entwickelt, um sowohl die technischen als auch die anwendungsbezogenen Aspekte der Arbeit zu berücksichtigen. Der Ausgangspunkt war das breite Feld der KI in Verbindung mit den alltäglichen Arbeitsaufgaben. Um die Arbeit jedoch spezifischer und fokussierter zu gestalten, wurde der Begriff ''AI'', erst durch ''NLP'' ersetzt. Um den Titel der Arbeit weiter zu präzisieren, wurde der Fokus auf das NLP-Tool ''Microsoft 365 Copilot'' gelegt. 

Darüber hinaus wurde der Titel so formuliert, dass er die Implementierung und Evaluierung von NLP in einem praktischen Geschäftsumfeld impliziert. Das Geschäftsumfeld wird noch konkretisiert mit dem Schwerpunkt auf den Schuheinzelhandel. Die Methodik der Expertenbefragung wird ebenfalls im Titel erwähnt, um die Erklärungsstrategie und den Untersuchungsansatz der Arbeit zu verdeutlichen. Um sicherzustellen, dass der gewählte Titel die Intention und den Inhalt der Arbeit optimal widerspiegelt, wurden verschiedene alternative Titel in Betracht gezogen:

\begin{enumerate}
    \item Die ersten Überlegungen zielten darauf ab, den Fokus auf den Schuheinzelhandel und die Anwendungsszenarien von KI zu legen:
    
    ''Die Rolle von künstlicher Intelligenz im Schuheinzelhandel: Bedeutung, Nutzen und Anwendungsszenarien''

    \item Um einen konkreteren Ansatz zu haben, wurde der Titel auf die Automatisierung manueller Prozesse mit KI in BI-Lösungen ausgerichtet:
    
    ''Automatisierung manueller Prozesse mit künstlicher Intelligenz und Business Intelligence Lösungen in einem Schuhhandelsunternehmen''

    \item Eine weitere Alternative war es, den Schwerpunkt auf die Anwendung auf NLPs in manuellen BI-Prozessen zu legen:
    
    ''Automatisierte Datenanalyse und Berichtserstellung mit Natural Language Processing in einem Schuhhandelsunternehmen''

    \item Weitere Alternativen für die Konzentration auf die Anwendung von LLMs in BI-Prozessen waren:
    
    ''Einsatz von Large Language Models zur Automatisierung von Geschäftsprozessen im Business Intelligence Kontext''

    ''Optimierung der Business Intelligence Landschaft durch Automatisierung manueller Prozesse mit Large Language Models''

    ''Automatisierung manueller Prozesse im Schuhhandelsunternehmen mit Large Language Models in Business Intelligence Lösungen''

    \item  Eine weitere Überlegung war es, dem Einfluss von NLP im Arbeitsalltag zu untersuchen:

    ''Analyse des Einflusses von Natural-Language-Processing-Tools auf betriebliche Arbeitsprozesse im Schuheinzelhandel: Eine Expertenbefragung''

\end{enumerate}

\clearpage